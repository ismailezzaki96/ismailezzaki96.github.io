\section{Les transitions de phase des trous noirs}

\begin{frame}{\underline{\secname} : Espace Asymptotiquement Plat}
\begin{block}{Trou noir de Schwarzschild :}
	
	\begin{columns}
		\begin{column}{0.5\linewidth}
		
	\begin{eqnarray*}
	T&=&\frac{1}{4\pi r_s}\,,\quad
	S=\frac{A}{4}=\pi r_s^2\,,\nonumber\\ G&=&\frac{r_s}{4}=\frac{1}{16\pi T}\,,\quad
	C_P=-2 \pi r_s^2<0\,,
	\end{eqnarray*}	
	\pause		
	
\begin{center}
	\textbf{Un trou noir de Schwarzschild est thermodynamiquement instable 	}
\end{center}

		\end{column}
		\begin{column}{0.5\linewidth}
			\begin{figure}[H]
				\begin{center}
					
					\includegraphics[width=\textwidth,height=140pt]{figures/GscF}
					
					\caption{$G(T)$ pour le trou noir de Schwarzschild}
				\end{center}
			\end{figure}
			
			
		\end{column}
	\end{columns}
\end{block}
\end{frame}


\begin{frame}{\underline{\secname} : Espace Asymptotiquement Plat}
\begin{block}{Trou noir de Reissner Nordstrom :}
	
	\begin{columns}
		\begin{column}{0.5\linewidth}
				\begin{eqnarray*}
		S&=& \pi r_s^2\,,\quad
		T = \frac{r_+^2-Q^2}{4\pi r_+^3}\,,\nonumber\\
		G&=&\frac{r_+^2+3Q^2}{4r_+}\,,\quad C_P=2\pi r_+^2\frac{r_+^2-Q^2}{3Q^2-r_+^2}\,,
			\end{eqnarray*}	
			
$C_p$ est positive  pour $\sqrt{3}|Q|> r_+>|Q|\,,$
	\pause		

\begin{center}
	\textbf{les petits trous noirs presque extrémaux fortement chargés sont thermodynamiquement préférée et stable}
\end{center}			
			
		\end{column}
		\begin{column}{0.5\linewidth}
			\begin{figure}[H]
				\begin{center}
					\includegraphics[width=\textwidth,height=140pt]{figures/Gkerrflat}
					
					\caption{$G(T)$ pour le trou noir de de Reissner Nordstrom}
				\end{center}
			\end{figure}
			
			
		\end{column}
	\end{columns}
\end{block}
\end{frame}



\begin{frame}{\underline{\secname} : Espace Asymptotiquement Plat}
\begin{block}{Trou noir de Kerr :}
	
	\begin{columns}
		\begin{column}{0.6\linewidth}
\begin{eqnarray*}
T&=&\frac{1}{2\pi}\bigg[\frac{r_+}{a^2+r_+^2}-\frac{1}{2r_+}\bigg]\,,\quad 
S=\pi(a^2+r_+^2)\,,\nonumber\\
H&=&\frac{r_+^2+a^2}{2r_+}\,,\quad  G=\frac{3a^2+r_+^2}{4r_+} \,,\nonumber\\
C_P&=&\frac{2\pi(r_+^2-a^2)(r_+^2+a^2)^2}{3a^4+6r_+^2a^2-r_+^4}
\end{eqnarray*}
$C_p$ est positive pour $\sqrt{3+2\sqrt{3}}|a|>r_+>|a|$

\pause
\begin{center}
	\textbf{les petits petits trous noirs à rotation rapide presque extrêmes sont thermodynamiquement préférée et stable}
\end{center}
		\end{column}
		\begin{column}{0.4\linewidth}
			\begin{figure}[H]
				\begin{center}
					\includegraphics[width=\textwidth,height=140pt]{figures/figure1}
				\end{center}
			\end{figure}
			
			
		\end{column}
	\end{columns}
\end{block}
\end{frame}







\begin{frame}{\underline{\secname} : Espace Asymptotiquement AdS}
\begin{block}{Trou noir de Schwarzschild AdS :}
les grandeurs thermodynamiques :
\begin{columns}
	\begin{column}{0.5\linewidth}
		la pression
$$P=-\dfrac{\Lambda}{8 \pi}=\dfrac{3}{8 \pi l^2}$$
La variable conjuguée à la pression
$$V= \left(\frac{\partial M}{\partial P} \right)_S=\frac{4}{3}\pi r_s^3$$
\pause
$\Rightarrow$ C’est le volume géométrique d’une sphère de rayon $r_s$ a 3 dimention
\pause
	\end{column}
\begin{column}{0.5\linewidth}
 			
 \begin{eqnarray*}
T&=&\frac{1}{4\pi r_s l^2}(l^2+3r_s^2)\,,\nonumber\\
S&=&\pi r_s^2\,,\quad 
H=\frac{r_s}{2}\Bigl(1+\frac{r_s^2}{l^2}\Bigr)\,,\nonumber\\ G&=&\frac{r_s}{4}\Bigl(1-\frac{r_s^2}{l^2}\Bigr)\,,\quad
C_P=2\pi r_s^2\frac{3r_s^2+l^2}{3r_s^2-l^2}\,,
 \end{eqnarray*}
	\end{column}
\end{columns}
\end{block}

\end{frame}

\begin{frame}{\underline{\secname} : Espace Asymptotiquement AdS}


\begin{center}
	\textbf{La stabilité et la chaleur spécifique}
\end{center}

\begin{columns}
	\begin{column}{0.5\linewidth}
La température minimale
$$\dfrac{\partial T}{\partial r_{+}}=0 \Leftrightarrow T_{min}=\dfrac{\sqrt{3}}{2\pi l}, \pause \Rightarrow$$
	\begin{itemize}
		\item $T < T_{min} $ la chaleur spécifique est négative, dans ce cas le système thermodynamique est instable .
		\item $T > T_{min} $ la chaleur spécifique est positive, donc c’est la région stable.
		
	\end{itemize}

		
\end{column}
\begin{column}{0.5\linewidth}
	
			\begin{figure}
		\begin{center}
			\includegraphics[width=\textwidth,height=140pt]{figures/path3285.png}
		\end{center}
	\end{figure}
			
\end{column}
\end{columns}

\end{frame}

\begin{frame}[t]{\underline{\secname} : Espace Asymptotiquement AdS}
\begin{center}
	\textbf{Transition de phase Hawking–Page:}
\end{center}

\begin{columns}
	\begin{column}{0.5\linewidth}
		 La transition de phase 
		$$G=0 \Leftrightarrow T_{HP}=\dfrac{1}{\pi l}.$$
	\begin{itemize} \setlength\itemsep{0em}
		\item  Pour $T < T_{min}$ , seulement la phase "rayonnement thermique pure" qui existe.
		\item  Pour $T_{min} < T < T_{HP}$ la phase "rayonnement thermique" est plus stable
		\item  Pour $T > T_{HP}$ la phase "large trou noir" c'est la plus stable
		
	\end{itemize}
		
		
	\end{column}
	\begin{column}{0.5\linewidth}
		\begin{figure}
			\begin{center}
				\includegraphics[width=\textwidth,height=140pt]{figures/g512.png}
				\label{Fig:Gschads}
			\end{center}
		\end{figure}
		
		
	\end{column}
\end{columns}
\end{frame}

\begin{frame}{\underline{\secname} : Espace Asymptotiquement AdS}

\begin{columns}
	\begin{column}{0.5\linewidth}
La ligne de coexistence des deux phases : rayonnement thermique/large trou noir s'écrit comme suit :
$$
P_{HP} =\frac{3 \pi}{8} T^2_{HP} 
$$
	\end{column}
\begin{column}{0.5\linewidth}
\begin{figure}
	\begin{center}
		\includegraphics[width=\textwidth,height=140pt]{figures/HPtransition}
	\end{center}
\end{figure}


\end{column}
\end{columns}
\end{frame}

%%%%%%%%%%%%%%%%%%%%%%%%%%%%%%%%%%%%%%%%%%%%


\begin{frame}{\underline{\secname} : Espace Asymptotiquement AdS}
\begin{block}{Trou noir de Reissner Nordstrom AdS :}
	les grandeurs thermodynamiques :
	
	\begin{eqnarray*}
T&=&\frac{1}{4\pi r_+^3 l^2}\Bigl(l^2(r_+^2-Q^2)+3r_+^4\Bigr)\,,\quad S=\pi r_+^2\,,\nonumber\\
V&=&\frac{4}{3}\pi r_+^3\,,\quad \Phi=\frac{Q}{r_+}\,,\quad 
G=\frac{l^2r_+^2-r_+^4+3Q^2l^2}{4l^2r_+}\,,\nonumber\\
C_P&=&2\pi r_+^2\frac{3r_+^4+l^2r_+^2-Q^2l^2}{3r_+^4-l^2r_+^2+3Q^2l^2}\,.\quad M= \frac{1}{2}\left(r_{+}-\frac{Q^{2}}{r_{+}}+\frac{8 \pi}{3} r_{+}^{3} P\right)
	\end{eqnarray*}


	
\end{block}

\end{frame}



\begin{frame}{\underline{\secname} : Espace Asymptotiquement AdS}

	\begin{columns}
		\begin{column}{0.6\linewidth}
\begin{itemize} \setlength\itemsep{0em}
	\item On peut écrire la pression comme:
	
	\begin{equation*}
	P=\frac{T}{\nu}-\frac{1}{2\pi \nu^2}+\frac{2Q^2}{\pi \nu^4}  \quad \text{avec: } \nu=2l_{p}^{2}r_{+}
	\end{equation*}
	\pause[2]
	$\Rightarrow$ Un équation d'état de \textbf{Van der Waals}:


\pause[3]
	\item	Le point critique:	
	\begin{equation*}
	\frac{\partial P}{\partial \nu}=0 \quad \text { et } \quad \frac{\partial^{2} P}{\partial \nu^{2}}=0
	\end{equation*}
			$$
			T_c=\frac{\sqrt{6}}{18\pi Q}\,,\quad \nonumber v_c=2\sqrt{6} Q\,,\quad \nonumber P_c=\frac{1}{96\pi Q^2}\,, 
			$$	
			
	\pause[4]				
\item la relation $\rho_c=P_c v_c/T_c=3/8$ est identique au cas de Van der Waals. 			
			
\end{itemize}
		\end{column}
		\begin{column}{0.4\linewidth}
			\pause[2]
			\begin{figure}[H]
				\begin{center}
			 
						\includegraphics[height=140pt,width=\textwidth]{figures/rP.pdf}
					
					\caption{Equation d'état d'un trou noir chargé AdS.}
				\end{center}
			\end{figure}
			
		\end{column}
	\end{columns}

\end{frame}


\begin{frame}{\underline{\secname} : Espace Asymptotiquement AdS}
\begin{center}
		\textbf{Transition de phase (SBH/LBH):}
\end{center}
\begin{columns}
	\begin{column}{0.5\linewidth}
		\begin{itemize} \setlength\itemsep{0.5em}
			\item  $P < P_c$ l’allure de $G$ montre une transition de phase du premier ordre
			\item  $P = P_c$ la transition de phase est de deuxième ordre 
			\item  $P > P_c$ pas de transition de phase
		\end{itemize}	
	\end{column}
	\begin{column}{0.5\linewidth}
\begin{figure}[H]
	\begin{center}

			\includegraphics[width=\textwidth,height=0.7\linewidth]{figures/GrnAdS.eps}
		
		\caption{ $G(T)$ d'un trou noir de RN-AdS.}
	\end{center}
\end{figure}
		
		
	\end{column}
\end{columns}

\end{frame}

%%%%%%%%%%%%%%%%%%%%%%%%%%%%%%



\begin{frame}{\underline{\secname} : Espace Asymptotiquement AdS}
\begin{block}{Trou noir de Kerr AdS :}
	les grandeurs thermodynamiques :
\begin{eqnarray*}
T&=&\frac{r_{+}}{4\pi(r_{+}^{2}+a^{2})}
\left(1+\frac{a^{2}}{l^{2}}+3\frac{r_{+}^{2}}{l^{2}}-\frac{a^{2}}{r_{+}^{2}}\right),\\
S&=&\frac{\pi (r_{+}^{2}+a^{2})}{\Xi},\quad 
\\
G&=&\frac{a^4 \left(r_{+}^2-l^{2}\right)+a^2\left(3l^{4}
	+2l^{2}r_{+}^2+3r_{+}^4\right)+l^{2} r_{+}^2
	\left(l^{2}-r_{+}^2\right)}{4 r_{+}\left(a^2-l^{2}\right)^2}.\quad 
\\
C_P &=& \frac{2 \pi l^4  \left(a^2+r_{+}^2\right)^2 \left(-a^2 l^2+\left(a^2+l^2\right) r_{+}^2
	+3 r_{+}^4\right)}{(l^2 -a^2) X},
\end{eqnarray*}
\end{block}
\end{frame}


%
%\begin{frame}{\underline{\secname} : Espace Asymptotiquement AdS}
%	
%	\begin{columns}
%		\begin{column}{0.5\linewidth}
% $$C_p(r)   \Rightarrow  C_p(p,s) $$
%
%Avec $p$ et $s$ les paramètres normalisés
%\begin{equation*}
%		p \equiv \frac{|a|}{r_{+}}, ~\mbox{and}~~ s \equiv \frac{l}{r_{+}}.
%\end{equation*}
%\begin{itemize}
%	\item 		Dans la région $\rom{1}$, $C_{P} <0$
%	\item	Dans la région $\rom{2}$,$C_{P} >0$
%	\item		Dans la région $\rom{3}$, il n'y a pas de solution physique parce que $|a| > l$.			
%\end{itemize}
%		\end{column}
%		\begin{column}{0.5\linewidth}
%		\begin{figure}[H]
%			\noindent\hfil\includegraphics[width=\linewidth,height=0.7\linewidth]{figures/figure14.eps}
%			\caption{Diagramme de phase des trous noirs de Kerr-AdS.}
%			\label{fig:heat_capacity_AdS}
%		\end{figure}
%
%			
%		\end{column}
%	\end{columns}
%
%\end{frame}




\begin{frame}{\underline{\secname} : Espace Asymptotiquement AdS}
	 Dans la limite du petit $J$ on obtient une valeur approximative de point critique \footnote{Shao-Wen Wei, Peng Cheng et Yu-Xiao Liu. “Analytical and exact critical phenomena ofd-dimensional singly spinning Kerr-AdS black holes”}:
	\begin{columns}
		\begin{column}{0.5\linewidth}
 	
			$$P_c \simeq  0.002857 \cdot J^{-1}$$
			$$V_c \simeq 6.047357 \cdot J^{1/2},$$
			$$T_c \simeq 115.796503 \sqrt[3]{J},$$

	\pause[3]		
	
	\begin{center}
		\textbf{Les résultats sont les mêmes que ceux du trou noir RN-AdS}	\\	
		{\color{green}Transition de phase (SBH/LBH)	}
	\end{center}

		\end{column}
		\begin{column}{0.5\linewidth}
			\pause[2]
	\begin{figure}[H]
		\begin{center}
			%	\rotatebox{-90}{
			\includegraphics[width=\textwidth,height=120pt]{figures/KerrGT_3b.eps}
			%	}
			\caption{ $G(T)$ d'un trou noir de Kerr-AdS.} \label{Fig:KKGT} 
		\end{center}
	\end{figure} 	
			
		\end{column}
	\end{columns}
\end{frame}


