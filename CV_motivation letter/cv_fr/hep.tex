%-------------------------------------------------------------------------------
%	SECTION TITLE
%-------------------------------------------------------------------------------

\lettersection{Projets 
}
%\lettersection{HEP projects}


%-------------------------------------------------------------------------------
%	CONTENT
%-------------------------------------------------------------------------------
\begin{cventries}
          \cventry
      {SiMULATiON D'UN iNTERFEROMETRE DE MACH ZEHNDER COMME CORONAGRAPHE
      } % Organization
    { } % Job title
    {} % Location
    {} % Date(s)
    {
      \begin{cvitems} % Description(s) of tasks/responsibilities
        \item {construire une simulation d'un interféromètre Mach Zehnder comme coronographe pour l'utiliser dans l'observatoire de l'Oukaimeden à Marrakech.}
        \end{cvitems}
    }
      \cventry
      {SiMULATiON D'éVéNEMENTS ATLAS à l'aide d'un réSEAU NEURAL GAN
      } % Organization
    { } % Job title
    {} % Location
    {} % Date(s)
    {
      \begin{cvitems} % Description(s) of tasks/responsibilities
        \item {Entraîner un réseau de neurones GAN à produire des événements Z → μμ dans des conditions qui reproduisent les collisions proton-proton au Large Hadron et le détecteur ATLAS
        }
        \end{cvitems}
    }
  \cventry
      {ANALYSES DU BOSON D'HiGGS DANS LES ÉTATS FiNAUx DE QUATRE LÉPTONS
      } % Organization
    {} % Job title
    {} % Location
    {} % Date(s)
    {
      \begin{cvitems} % Description(s) of tasks/responsibilities
        \item {Reproduire l'analyse de la découverte du boson de Higgs en utilisant les ensembles de données de 2011 et 2012, dans les états finaux à quatre leptons.}
        \end{cvitems}
    }
%          \cventry
%      {Particle info python library} % Organization
%    { } % Job title
%    {} % Location
%    {} % Date(s)
%    {
%      \begin{cvitems} % Description(s) of tasks/responsibilities
%        \item {provide a simple interface in python to the Particle Data Group (PDG) with extended particle information.}
%        \end{cvitems}
%    }
      \cventry
      {boson de higgs \& apprentissage automatique
       } % Organization
    { } % Job title
    {} % Location
    {} % Date(s)
    {
      \begin{cvitems} % Description(s) of tasks/responsibilities
        \item {Explorer le potentiel des méthodes avancées d'apprentissage automatique pour améliorer la signification de la découverte de l'expérience ATLAS.}
        \end{cvitems}
    }
    
%---------------------------------------------------------
 

%---------------------------------------------------------
 

%---------------------------------------------------------
 
%---------------------------------------------------------
 

%---------------------------------------------------------
\end{cventries}
