		 \lettersection{About me}

%//////////////
% astronomy
%I graduated last year with a Master’s degree in ”high energy and computational physics” from Cadi Ayyad University in Morocco.


I graduated last year with a Master’s degree in ”high energy and astronomy and computational physics” from Cadi Ayyad University in Morocco.


%///////////
%I feel comfortable processing several challenging things at one time. This ability helps me to be successful in study (my current GPA is 4.88 out of 5.00) and acquire extra-curricular knowledge. In order to better understand optical effects observed in my experimental thesis work, I decided to broaden my knowledge related to the studied problem. Hence I devoted my spare time for mastering new-to-me discipline: numerical methods for simulation of amplification and decay of holographic gratings in photopolymers. In order to grasp this knowledge I participated in another research group of my Department led by Professor Veniaminov who specialises on slow diffusion processes. Familiarising myself with his research, solving proposed computational tasks, I achieved a dipper insight into phenomenon investigated in my experimental activity. I think that this experience proves my ability to organize work effectively and achieve goals pursuing clear vision of final result.

I feel comfortable processing several challenging things at once. This ability helps me to be successful in my study ( \nth{2}  in my promotion of master) and gain extra-curricular knowledge in the computer science field (about 5 years of experience with programming).
%//////////////

Since my childhood, I’ve always been interested in discovering how things work so I choose to study physics, like many kids, I've been dreaming of winning a Nobel prize in physics but when I grew up I changed to a realistic dream of becoming an academic researcher

My interest in programming began around five years ago when I first came into contact with computer coding in school while creating a small website. From there on, I acquired  extra-curricular knowledge in programming in C++ and python and also FORTRAN and also I learned skills of data analysis and machine learning 
%/////////////////
%My interest in programming began around five years ago when I first came into contact with computer coding in school while creating a small website. From there on, I was captivated. I started working on small projects of my own, creating other websites and a simple calculator. I have learned small pieces of code from several languages through the Harvard lectures that are available online free and through others, working on computer code is my way of being creative since I find it difficult to be creative in more common ways such as drawing. Every time I write a piece of computer code, I feel that I have created a piece of art,  and reading computer code feels like looking at a piece of art.
%//////////////


My correct research interest is in the area of data analysis in experimental particle physics and I worked on many high energy projects related to data analysis in high energy physics and in computational physics in general you can find more information about them in my CV

%My correct research interest is in the area of computational physics and I worked on many projects related to it, you can find more information about them in my CV

%My primary field of research interest is computational physics. More precisely, I am interested in problems related to particle physics. 

%Because of the intrinsic interdisciplinary nature of computational physics, my interests often overlap with other areas of computer science such as machine learning, algorithms, and data analysis. 

%In the following, I will briefly summarize some of my research projects related to computational and particle physics.



%As regards my future physics interests, I would like to continue in computational physics and made more software that make the life of physicist easier, also after my experience with machine learning, I want to focus on the usage of its features in my programs since it is quickly providing new powerful tools for physicists either in experiments or simulations. 

 
